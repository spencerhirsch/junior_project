%%%%%%%%%%%%%%%%%%%%%%%%%%%%%%%%%%%%%%%%%%%%%%%%%%%%%%%%%%%%%%%%%%%%%%%%%%%%%%%%%%%%%%%%%
%%%%%%%%%%%%%%%%%%%%%%%%%%%%%%%%%%%%%%%%%%%%%%%%%%%%%%%%%%%%%%%%%%%%%%%%%%%%%%%%%%%%%%%%%
%%%%%%%%% LaTeX beamer metropolis slides template for the High Energy Physics  %%%%%%%%%%
%%%%%%%%% Group (CMS) at the Florida Institute of Technology. Uses the         %%%%%%%%%%
%%%%%%%%% metropolis theme by Matthias Vogelgesang.			       %%%%%%%%%%
%%%%%%%%% https://ctan.org/pkg/beamertheme-metropolis?lang=en                  %%%%%%%%%%
%%%%%%%%%%%%%%%%%%%%%%%%%%%%%%%%%% Stephen D. Butalla %%%%%%%%%%%%%%%%%%%%%%%%%%%%%%%%%%%
%%%%%%%%%%%%%%%%%%%%%%%%%%%%%%%%%%%%%% 2021/02/01 %%%%%%%%%%%%%%%%%%%%%%%%%%%%%%%%%%%%%%% 
%%%%%%%%%%%%%%%%%%%%%%%%%%%%%%%%%%%%%%%%%%%%%%%%%%%%%%%%%%%%%%%%%%%%%%%%%%%%%%%%%%%%%%%%%

\documentclass[10pt]{beamer}
\usetheme{metropolis}
\usepackage{xcolor}
\usepackage{booktabs}
\usepackage[scale=2]{ccicons}
\usepackage{graphicx}
\usepackage{adjustbox}
\usepackage{transparent}
\usepackage{eso-pic}
\graphicspath{	{images/}	} % Change to your graphics path
\usepackage{amsmath}
\usepackage{url}
\usepackage{hyperref}
\usepackage{listings}
\lstset{language=Python,
    basicstyle=\ttfamily\bfseries,
    commentstyle=\color{red}\itshape,
  showstringspaces=false,
  keywordstyle=\color{blue}\bfseries}
\setbeamertemplate{bibliography item}[text]

\DeclareMathOperator\erf{erf} % define the error function for math mode


\definecolor{FTRed}{RGB}{114, 116, 156} % defines Florida Tech red (crimson)_ color
\setbeamercolor{progress bar}{fg=FTRed} % color progress 
\setbeamercolor{frametitle}{fg=white,bg=FTRed} % set top bar of slide to FTRed

%Hides section slides
\metroset{sectionpage=none}

\usepackage{pgfplots}
\usepgfplotslibrary{dateplot}

\usepackage{xspace}
\newcommand{\themename}{\textbf{\textsc{metropolis}}\xspace}

 \hypersetup{
     colorlinks=true,
     linkcolor=blue,
     filecolor=blue,
     citecolor = black,      
     urlcolor=FTRed,
 }


\title{Milestone 4 Presentation}
\subtitle{My Takeaway from Junior Project}
\date{\today}
\author{Spencer Hirsch}
\institute{Florida Institute of Technology}
% \titlegraphic{
% 	\begin{picture}(0,0)
%     \put(70,-175){\makebox(0,0)[rt]{\includegraphics[height=3cm]{FloridaTechNewLogo.png}}}
%   \end{picture}
% \begin{picture}(0,0)
%     \put(320,-160){\makebox(0,0)[rt]{\includegraphics[height=3cm]{CMSlines.png}}}
%   \end{picture}
%   \begin{picture}(0,0)
%     \put(320,10){\makebox(0,0)[rt]{\includegraphics[height=2.2cm]{cern_logo.png}}}
%   \end{picture}
  
 
 %}

% Add footer
\setbeamertemplate{footline}[text line]{%
  \parbox{\linewidth}{\vspace{-0.15cm}\hspace{4cm} S.\ Hirsch  -- ``Milestone 4'' -- \today \hfill\insertpagenumber}}


\begin{document}
\maketitle
%Adds Florida Tech logo to each page after this command
% \addtobeamertemplate{frametitle}{}{%
% \begin{tikzpicture}[remember picture,overlay]
% \node[anchor=south west, yshift=-4pt] at (current page.south west) {\includegraphics[height=1.2cm]{FloridaTechNewLogo}};
% \end{tikzpicture}}
% \addtobeamertemplate{frametitle}{}{%
% \begin{tikzpicture}[remember picture,overlay]
% \node[anchor=north east,yshift=2pt,xshift=1.5pt] at (current page.north east) {\includegraphics[height=0.8cm]{CMS-Color-Var1.pdf}};
% \end{tikzpicture}}
% \addtobeamertemplate{frametitle}{}{%
% \begin{picture}(0,0)
%     \put(305,29.2){\makebox(0,0)[rt]{\includegraphics[height=0.8cm]{cern_logo_1}}}
%   \end{picture}}
  
% If you want a table of contents,
% uncomment lines 93--97 for default
% table of contents, or 101--112 for
% the two-column table of contents

%Default table of contents
%\begin{frame}{Table of contents}
 %\setbeamertemplate{section in toc}[sections numbered]
  %\tableofcontents[hideallsubsections]
%\end{frame}

% Two column table of contents
%\begin{frame}{Table of contents}
%        \begin{columns}
%            \begin{column}{.5\textwidth}
%             \setbeamertemplate{section in toc}[sections numbered]
  %              \tableofcontents[sections=1- 6]
  %              
     %       \end{column}
        %    \begin{column}{.5\textwidth}
           %  \setbeamertemplate{section in toc}[sections numbered]
           %     \tableofcontents[sections=7-11]
           % \end{column}
       % \end{columns}
   % \end{frame}




% \begin{frame}[fragile]{Frame with columns}
% \vspace{-0.5cm}
% \begin{columns}
% \begin{column}{0.5\textwidth}
% \begin{itemize}\scriptsize{
% \itemsep0em
% \item Column 0}
% \end{itemize}
% \end{column}
% \begin{column}{0.5\textwidth}
% \begin{itemize}\scriptsize{
% \itemsep0em
% \item Column 1}
% \end{itemize}
% \end{column}
% \end{columns}
% \end{frame}

% \begin{frame}[fragile]{Frame with figure}

% \center
% \includegraphics[width=3cm]{CMS-Color-Var1.pdf}\\
% \scriptsize{The CMS logo.}\\


% \end{frame}

% \begin{frame}[fragile]{Frame with equation}

% $$i\gamma^{\mu}\partial_{\mu}\psi-\psi=0$$
% \end{frame}

\begin{frame}[fragile]{Introduction}
  \begin{itemize}
    \item The four milestones are all intended to teach us different aspects of preparing for our Senior Projects
    \begin{enumerate}
      \item Interview a Senior Project Group
      \item Communication and methods for groups
      \item Selecting a project
      \item Present findings
    \end{enumerate}
  \end{itemize}
\end{frame}

\begin{frame}[fragile]{Milestone 1: Interview}
  \begin{itemize}
    \item I began thinking about potential projects
    \item Got better insight on what to expect from a senior project
    \item Got to see first hand some issues that the group ran into
  \end{itemize}
\end{frame}

\begin{frame}[fragile]{Milestone 2: Communication}
  \begin{itemize}
    \item Looked into different software applications to host a project
    \item Looked into different communication applications for group work
    \item Decided on using GitHub for project collaboration
    \item Discord will most likely be the application that is used for group communication
  \end{itemize}
\end{frame}

\begin{frame}[fragile]{Milestone 3: Preparation}
  \begin{itemize}
    \item Reached out to professors on potential projects
    \item Began thinking more about what my project will look like
      \begin{itemize}
        \item Begin working with a professor on a new project
        \item Continuing working with my current research group
      \end{itemize}
  \end{itemize}
\end{frame}

\begin{frame}[fragile]{Conclusion}
  \begin{itemize}
    \item Junior Project has been beneficial in getting the ball rolling for Senior Project Ideas
    \item Got insight on what the work looks like while working on a project
    \item Reached out to some professors and got some ideas for projects I may be interested in working on
    \item Curently deciding if I want to join a professor on a project or if I want to make my own
  \end{itemize}
\end{frame}

\end{document}



